% Setting up the document class for a clean, professional look
\documentclass[11pt, a4paper]{article}

% Including essential packages for formatting, mathematics, images, pseudocode, and tables
\usepackage[utf8]{inputenc}
\usepackage[T1]{fontenc}
\usepackage{amsmath, amsthm, amssymb}
\usepackage{geometry}
\usepackage{titling}
\usepackage{enumitem}
\usepackage{xcolor}
\usepackage{hyperref}
\usepackage{fancyhdr}
\usepackage{parskip}
\usepackage{graphicx}
\usepackage{algorithm}
\usepackage{algpseudocode}
\usepackage{booktabs} % For professional-looking tables

% Configuring page geometry for balanced margins
\geometry{margin=1in}

% Defining custom theorem-like environments
\theoremstyle{plain}
\newtheorem{theorem}{Theorem}[section]
\newtheorem{lemma}[theorem]{Lemma}
\newtheorem{proposition}[theorem]{Proposition}
\newtheorem{corollary}[theorem]{Corollary}

\theoremstyle{definition}
\newtheorem{definition}[theorem]{Definition}
\newtheorem{property}[theorem]{Property}
\newtheorem{example}[theorem]{Example}
\newtheorem{remark}[theorem]{Remark}

% Ensuring all environments use black, upright font
\let\oldtheorem\theorem
\renewcommand{\theorem}{\oldtheorem\upshape\color{black}}
\let\oldlemma\lemma
\renewcommand{\lemma}{\oldlemma\upshape\color{black}}
\let\oldproposition\proposition
\renewcommand{\proposition}{\oldproposition\upshape\color{black}}
\let\oldcorollary\corollary
\renewcommand{\corollary}{\oldcorollary\upshape\color{black}}
\let\olddefinition\definition
\renewcommand{\definition}{\olddefinition\upshape\color{black}}
\let\oldproperty\property
\renewcommand{\property}{\oldproperty\upshape\color{black}}
\let\oldexample\example
\renewcommand{\example}{\oldexample\upshape\color{black}}
\let\oldremark\remark
\renewcommand{\remark}{\oldremark\upshape\color{black}}

% Setting up header and footer
\pagestyle{fancy}
\fancyhf{}
\fancyhead[L]{\thetitle} % Display document title in header
\fancyhead[R]{\thepage}

% Configuring hyperlinks
\hypersetup{
	colorlinks=true,
	linkcolor=blue,
	urlcolor=blue,
	citecolor=blue
}

% Setting up title and author
\setlength{\parindent}{0pt}
\setlength{\parskip}{1em}
\author{Your Name}
\title{Mathematical Notes}

% Configuring font package (loaded last as per guidelines)
\usepackage{newtxtext, newtxmath}

\begin{document}
	
	% Creating title page with summary at the bottom and revision date
	\begin{titlepage}
		\centering
		\vspace*{2cm}
		{\Huge\bfseries \thetitle \par}
		\vspace{1cm}
		{\Large \theauthor \par}
		\vspace{0.5cm}
		{\large Last revised at 2025/6/10 \par}
		\vspace{3.5cm} % Adjusted vertical space to maintain summary position
		{\large\bfseries Summary \par}
		\vspace{0.5cm}
		This document provides a structured template for mathematical notes, including definitions, theorems, lemmas, properties, examples, pseudocode, tables, and images. The content is organized to facilitate clear understanding and reference, with a focus on mathematical rigor and aesthetic presentation.
	\end{titlepage}
	
	% Adding main content sections
	\section{Introduction}
	This section introduces the main topics covered in the notes. You can provide an overview of the subject matter here.
	
	\section{Definitions and Properties}
	\begin{definition}
		A group is a set \( G \) equipped with a binary operation \( \cdot \) that satisfies the following axioms:
		\begin{enumerate}[label=\roman*.]
			\item Closure: For all \( a, b \in G \), \( a \cdot b \in G \).
			\item Associativity: For all \( a, b, c \in G \), \( (a \cdot b) \cdot c = a \cdot (b \cdot c) \).
			\item Identity: There exists an element \( e \in G \) such that for all \( a \in G \), \( e \cdot a = a \cdot e = a \).
			\item Inverse: For each \( a \in G \), there exists an element \( b \in G \) such that \( a \cdot b = b \cdot a = e \).
		\end{enumerate}
	\end{definition}
	
	\begin{property}
		Every group has a unique identity element.
	\end{property}
	
	\section{Theorems and Proofs}
	\begin{theorem}
		Let \( G \) be a group. Then the identity element \( e \) in \( G \) is unique.
	\end{theorem}
	\begin{proof}
		Suppose \( e \) and \( e' \) are both identity elements in \( G \). Then, for all \( a \in G \), we have \( e \cdot a = a \) and \( a \cdot e' = a \). Consider \( e \cdot e' \). Since \( e \) is an identity, \( e \cdot e' = e' \). Since \( e' \) is an identity, \( e \cdot e' = e \). Thus, \( e = e' \), proving the identity is unique.
	\end{proof}
	
	\begin{lemma}
		In any group \( G \), the inverse of each element is unique.
	\end{lemma}
	\begin{proof}
		Let \( a \in G \) have two inverses \( b \) and \( c \). Then \( a \cdot b = e \) and \( a \cdot c = e \). Multiply both sides of \( a \cdot b = e \) by \( c \) on the right: \( (a \cdot b) \cdot c = e \cdot c = c \). By associativity, \( a \cdot (b \cdot c) = c \). Since \( a \cdot c = e \), we have \( a \cdot (b \cdot c) = e \). Thus, \( b \cdot c \) is an inverse of \( a \). Since \( b \) is an inverse, \( b \cdot c = b \). Multiply both sides by the inverse of \( b \): \( c = b \), so the inverse is unique.
	\end{proof}
	
	\begin{example}
		Consider the group \( (\mathbb{Z}, +) \). The identity element is \( 0 \), and the inverse of any integer \( a \) is \( -a \), since \( a + (-a) = 0 \).
	\end{example}
	
	\begin{remark}
		The concepts introduced here can be extended to other algebraic structures, such as rings and fields.
	\end{remark}
	
	\section{Pseudocode Example}
	\begin{algorithm}
		\caption{Group Element Inverse Check}
		\begin{algorithmic}[1]
			\Procedure{FindInverse}{$a, G, \cdot, e$}
			\For{each $b \in G$}
			\If{$a \cdot b = e$ and $b \cdot a = e$}
			\State \Return $b$
			\EndIf
			\EndFor
			\State \Return \text{None}
			\EndProcedure
		\end{algorithmic}
	\end{algorithm}
	
	\section{Table Example}
	\begin{table}[h]
		\centering
		\caption{Properties of Common Groups}
		\begin{tabular}{lcc}
			\toprule
			Group & Identity Element & Inverse of $a$ \\
			\midrule
			\( (\mathbb{Z}, +) \) & 0 & $-a$ \\
			\( (\mathbb{R}^*, \cdot) \) & 1 & $1/a$ \\
			\( (S_n, \circ) \) & Identity permutation & Inverse permutation \\
			\bottomrule
		\end{tabular}
		\label{tab:group_properties}
	\end{table}
	
	\section{Image Inclusion Examples}
	% Single image example
	\begin{figure}[h]
		\centering
		% Replace 'example-image' with your image file name (e.g., 'group_diagram.png')
		\includegraphics[width=0.5\textwidth]{example-image}
		\caption{A diagram illustrating a group structure.}
		\label{fig:group_diagram}
	\end{figure}
	
	% Two images side by side
	\begin{figure}[h]
		\centering
		\begin{minipage}{0.45\textwidth}
			\centering
			\includegraphics[width=\textwidth]{example-image-a}
			\caption{First group diagram.}
			\label{fig:group_diagram_a}
		\end{minipage}\hfill
		\begin{minipage}{0.45\textwidth}
			\centering
			\includegraphics[width=\textwidth]{example-image-b}
			\caption{Second group diagram.}
			\label{fig:group_diagram_b}
		\end{minipage}
	\end{figure}
	
\end{document}