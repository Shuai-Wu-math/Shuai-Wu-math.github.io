% Setting up the document class and basic layout
\documentclass[11pt, a4paper]{article}
\usepackage[utf8]{inputenc}
\usepackage[T1]{fontenc}
\usepackage{geometry}
\geometry{margin=1in}

% Including essential packages for formatting and content
\usepackage{amsmath, amssymb, mathtools}
\usepackage{graphicx}
\usepackage{subcaption}
\usepackage{booktabs}
\usepackage{array}
\usepackage{enumitem}
\usepackage{listings}
\usepackage{xcolor}
\usepackage{algorithm}
\usepackage{algpseudocode}

% Defining code listing styles with enhanced indentation
\lstset{
	basicstyle=\ttfamily\small,
	breaklines=true,
	breakatwhitespace=true,
	breakindent=10pt,
	breakautoindent=true,
	frame=single,
	numbers=left,
	numberstyle=\tiny,
	keywordstyle=\color{blue},
	stringstyle=\color{red},
	commentstyle=\color{gray},
	showstringspaces=false,
	tabsize=4,
	keepspaces=true,
	columns=flexible
}

% Defining language-specific styles
\lstdefinelanguage{Python}{
	keywords={def, return, if, else, for, while, in, print, import},
	morecomment=[l]{\#}
}
\lstdefinelanguage{MATLAB}{
	keywords={function, end, if, else, for, while},
	morecomment=[l]{\%}
}
\lstdefinelanguage{C}{
	keywords={int, void, return, if, else, for, while},
	morecomment=[l]{//}
}

% Setting up font package last
\usepackage{lmodern}

% Title and header setup
\title{Homework Assignment Template}
\author{Your Name}
\date{\today}

\begin{document}
	
	\maketitle
	
	\section*{Homework Assignment}
	
	% Question 1
	\subsection*{Question 1: Basic Arithmetic Function}
	Implement a function that calculates the sum of squares of numbers from 1 to $n$.
	
	\subsubsection*{Solution}
	\begin{itemize}
		\item \textbf{Python Code:}
		\begin{lstlisting}[language=Python]
def sum_of_squares(n):
total = 0
for i in range(1, n + 1):
total += i ** 2
return total
		\end{lstlisting}
		\item \textbf{MATLAB Code:}
		\begin{lstlisting}[language=MATLAB]
function result = sumOfSquares(n)
result = sum((1:n).^2);
end
		\end{lstlisting}
		\item \textbf{C Code:}
		\begin{lstlisting}[language=C]
int sumOfSquares(int n) {
	int total = 0;
	for (int i = 1; i <= n; i++) {
		total += i * i;
	}
	return total;
}
		\end{lstlisting}
	\end{itemize}
	
	% Question 2
	\subsection*{Question 2: Matrix Operations}
	Write a program to compute the transpose of a given matrix.
	
	\subsubsection*{Solution}
	\begin{itemize}
		\item \textbf{Python Code:}
		\begin{lstlisting}[language=Python]
def transpose_matrix(matrix):
return [[matrix[j][i] for j in range(len(matrix))] for i in range(len(matrix[0]))]
		\end{lstlisting}
		\item \textbf{MATLAB Code:}
		\begin{lstlisting}[language=MATLAB]
function result = transposeMatrix(matrix)
result = matrix';
end
		\end{lstlisting}
		\item \textbf{C Code:}
		\begin{lstlisting}[language=C]
void transposeMatrix(int rows, int cols, int matrix[rows][cols], int result[cols][rows]) {
	for (int i = 0; i < rows; i++) {
		for (int j = 0; j < cols; j++) {
			result[j][i] = matrix[i][j];
		}
	}
}
		\end{lstlisting}
	\end{itemize}
	
	% Question 3 with Table Example
	\subsection*{Question 3: Data Analysis}
	Create a table to summarize the performance metrics of an algorithm (e.g., time and accuracy).
	
	\subsubsection*{Solution}
	Below is a sample table summarizing performance metrics:
	\begin{table}[h]
		\centering
		\caption{Algorithm Performance Metrics}
		\begin{tabular}{lcc}
			\toprule
			\textbf{Algorithm} & \textbf{Execution Time (s)} & \textbf{Accuracy (\%)} \\
			\midrule
			Algorithm A & 0.25 & 95.3 \\
			Algorithm B & 0.30 & 92.1 \\
			Algorithm C & 0.18 & 97.2 \\
			\bottomrule
		\end{tabular}
	\end{table}
	
	% Question 4 with Image Example
	\subsection*{Question 4: Image Processing}
	Describe an algorithm to process two images side by side.
	
	\subsubsection*{Solution}
	Below are two sample images displayed side by side:
	\begin{figure}[h]
		\centering
		\begin{subfigure}{0.45\textwidth}
			\centering
			\includegraphics[width=\linewidth]{example-image-a}
			\caption{Image A}
		\end{subfigure}
		\hfill
		\begin{subfigure}{0.45\textwidth}
			\centering
			\includegraphics[width=\linewidth]{example-image-b}
			\caption{Image B}
		\end{subfigure}
		\caption{Side-by-side comparison of two images.}
	\end{figure}
	
	\begin{itemize}
		\item \textbf{Python Code (Image Processing Placeholder):}
		\begin{lstlisting}[language=Python]
import cv2
def process_image(image):
# Placeholder for image processing
return cv2.cvtColor(image, cv2.COLOR_BGR2GRAY)
		\end{lstlisting}
	\end{itemize}
	
	% Question 5 with Pseudocode Example
	\subsection*{Question 5: Sorting Algorithm}
	Implement a sorting algorithm and provide its pseudocode.
	
	\subsubsection*{Solution}
	Below is the pseudocode for QuickSort:
	\begin{algorithm}
		\caption{QuickSort Algorithm}
		\begin{algorithmic}
			\Procedure{QuickSort}{array, low, high}
			\If{low < high}
			\State pivotIndex $\gets$ \Call{Partition}{array, low, high}
			\State \Call{QuickSort}{array, low, pivotIndex - 1}
			\State \Call{QuickSort}{array, pivotIndex + 1, high}
			\EndIf
			\EndProcedure
			\Procedure{Partition}{array, low, high}
			\State pivot $\gets$ array[high]
			\State i $\gets$ low - 1
			\For{j $\gets$ low \textbf{to} high - 1}
			\If{array[j] $\leq$ pivot}
			\State i $\gets$ i + 1
			\State Swap array[i] and array[j]
			\EndIf
			\EndFor
			\State Swap array[i využivané + 1] and array[high]
			\State \Return i + 1
			\EndProcedure
		\end{algorithmic}
	\end{algorithm}
	
	\begin{itemize}
		\item \textbf{Python Code:}
		\begin{lstlisting}[language=Python]
def quicksort(arr, low, high):
if low < high:
pi = partition(arr, low, high)
quicksort(arr, low, pi - 1)
quicksort(arr, pi + 1, high)

def partition(arr, low, high):
pivot = arr[high]
i = low - 1
for j in range(low, high):
if arr[j] <= pivot:
i += 1
arr[i], arr[j] = arr[j], arr[i]
arr[i + 1], arr[high] = arr[high], arr[i + 1]
return i + 1
		\end{lstlisting}
	\end{itemize}
	
\end{document}